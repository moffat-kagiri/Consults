% Options for packages loaded elsewhere
\PassOptionsToPackage{unicode}{hyperref}
\PassOptionsToPackage{hyphens}{url}
%
\documentclass[
  12pt,
]{article}
\usepackage{lmodern}
\usepackage{amssymb,amsmath}
\usepackage{ifxetex,ifluatex}
\ifnum 0\ifxetex 1\fi\ifluatex 1\fi=0 % if pdftex
  \usepackage[T1]{fontenc}
  \usepackage[utf8]{inputenc}
  \usepackage{textcomp} % provide euro and other symbols
\else % if luatex or xetex
  \usepackage{unicode-math}
  \defaultfontfeatures{Scale=MatchLowercase}
  \defaultfontfeatures[\rmfamily]{Ligatures=TeX,Scale=1}
\fi
% Use upquote if available, for straight quotes in verbatim environments
\IfFileExists{upquote.sty}{\usepackage{upquote}}{}
\IfFileExists{microtype.sty}{% use microtype if available
  \usepackage[]{microtype}
  \UseMicrotypeSet[protrusion]{basicmath} % disable protrusion for tt fonts
}{}
\makeatletter
\@ifundefined{KOMAClassName}{% if non-KOMA class
  \IfFileExists{parskip.sty}{%
    \usepackage{parskip}
  }{% else
    \setlength{\parindent}{0pt}
    \setlength{\parskip}{6pt plus 2pt minus 1pt}}
}{% if KOMA class
  \KOMAoptions{parskip=half}}
\makeatother
\usepackage{xcolor}
\IfFileExists{xurl.sty}{\usepackage{xurl}}{} % add URL line breaks if available
\IfFileExists{bookmark.sty}{\usepackage{bookmark}}{\usepackage{hyperref}}
\hypersetup{
  pdftitle={Federal Presidential Elections 2020},
  pdfauthor={Langwen Guan, Yuhang Ju, Zike Peng},
  hidelinks,
  pdfcreator={LaTeX via pandoc}}
\urlstyle{same} % disable monospaced font for URLs
\usepackage[margin = 1in]{geometry}
\usepackage{color}
\usepackage{fancyvrb}
\newcommand{\VerbBar}{|}
\newcommand{\VERB}{\Verb[commandchars=\\\{\}]}
\DefineVerbatimEnvironment{Highlighting}{Verbatim}{commandchars=\\\{\}}
% Add ',fontsize=\small' for more characters per line
\usepackage{framed}
\definecolor{shadecolor}{RGB}{248,248,248}
\newenvironment{Shaded}{\begin{snugshade}}{\end{snugshade}}
\newcommand{\AlertTok}[1]{\textcolor[rgb]{0.94,0.16,0.16}{#1}}
\newcommand{\AnnotationTok}[1]{\textcolor[rgb]{0.56,0.35,0.01}{\textbf{\textit{#1}}}}
\newcommand{\AttributeTok}[1]{\textcolor[rgb]{0.77,0.63,0.00}{#1}}
\newcommand{\BaseNTok}[1]{\textcolor[rgb]{0.00,0.00,0.81}{#1}}
\newcommand{\BuiltInTok}[1]{#1}
\newcommand{\CharTok}[1]{\textcolor[rgb]{0.31,0.60,0.02}{#1}}
\newcommand{\CommentTok}[1]{\textcolor[rgb]{0.56,0.35,0.01}{\textit{#1}}}
\newcommand{\CommentVarTok}[1]{\textcolor[rgb]{0.56,0.35,0.01}{\textbf{\textit{#1}}}}
\newcommand{\ConstantTok}[1]{\textcolor[rgb]{0.00,0.00,0.00}{#1}}
\newcommand{\ControlFlowTok}[1]{\textcolor[rgb]{0.13,0.29,0.53}{\textbf{#1}}}
\newcommand{\DataTypeTok}[1]{\textcolor[rgb]{0.13,0.29,0.53}{#1}}
\newcommand{\DecValTok}[1]{\textcolor[rgb]{0.00,0.00,0.81}{#1}}
\newcommand{\DocumentationTok}[1]{\textcolor[rgb]{0.56,0.35,0.01}{\textbf{\textit{#1}}}}
\newcommand{\ErrorTok}[1]{\textcolor[rgb]{0.64,0.00,0.00}{\textbf{#1}}}
\newcommand{\ExtensionTok}[1]{#1}
\newcommand{\FloatTok}[1]{\textcolor[rgb]{0.00,0.00,0.81}{#1}}
\newcommand{\FunctionTok}[1]{\textcolor[rgb]{0.00,0.00,0.00}{#1}}
\newcommand{\ImportTok}[1]{#1}
\newcommand{\InformationTok}[1]{\textcolor[rgb]{0.56,0.35,0.01}{\textbf{\textit{#1}}}}
\newcommand{\KeywordTok}[1]{\textcolor[rgb]{0.13,0.29,0.53}{\textbf{#1}}}
\newcommand{\NormalTok}[1]{#1}
\newcommand{\OperatorTok}[1]{\textcolor[rgb]{0.81,0.36,0.00}{\textbf{#1}}}
\newcommand{\OtherTok}[1]{\textcolor[rgb]{0.56,0.35,0.01}{#1}}
\newcommand{\PreprocessorTok}[1]{\textcolor[rgb]{0.56,0.35,0.01}{\textit{#1}}}
\newcommand{\RegionMarkerTok}[1]{#1}
\newcommand{\SpecialCharTok}[1]{\textcolor[rgb]{0.00,0.00,0.00}{#1}}
\newcommand{\SpecialStringTok}[1]{\textcolor[rgb]{0.31,0.60,0.02}{#1}}
\newcommand{\StringTok}[1]{\textcolor[rgb]{0.31,0.60,0.02}{#1}}
\newcommand{\VariableTok}[1]{\textcolor[rgb]{0.00,0.00,0.00}{#1}}
\newcommand{\VerbatimStringTok}[1]{\textcolor[rgb]{0.31,0.60,0.02}{#1}}
\newcommand{\WarningTok}[1]{\textcolor[rgb]{0.56,0.35,0.01}{\textbf{\textit{#1}}}}
\usepackage{graphicx,grffile}
\makeatletter
\def\maxwidth{\ifdim\Gin@nat@width>\linewidth\linewidth\else\Gin@nat@width\fi}
\def\maxheight{\ifdim\Gin@nat@height>\textheight\textheight\else\Gin@nat@height\fi}
\makeatother
% Scale images if necessary, so that they will not overflow the page
% margins by default, and it is still possible to overwrite the defaults
% using explicit options in \includegraphics[width, height, ...]{}
\setkeys{Gin}{width=\maxwidth,height=\maxheight,keepaspectratio}
% Set default figure placement to htbp
\makeatletter
\def\fps@figure{htbp}
\makeatother
\setlength{\emergencystretch}{3em} % prevent overfull lines
\providecommand{\tightlist}{%
  \setlength{\itemsep}{0pt}\setlength{\parskip}{0pt}}
\setcounter{secnumdepth}{-\maxdimen} % remove section numbering

\begin{document}

usepackage{setspace} doublespacing


\title{Federal Presidential Elections 2020}
\author{Langwen Guan, Yuhang Ju, Zike Peng}
\date{11/2/2020}

\maketitle

\newpage

\hypertarget{introduction}{%
\section{Introduction}\label{introduction}}

Elections in the United States of America are determined by a number of
factors. Voters are primarily swayed by their identities, beliefs, and
contexts. Since these issues are associated to their individual
demographic classes, one can use a regression model to predict the
outcome of the 2020 federal elections. We came up with a logistic
regression model using the Nationscape dataset as a sample population.
This dataset describes a number of demographic factors in the United
States which affect the political affiliations of voters. The dataset
enabled us to develop an accurate model to predict the outcome of this
year's elections.

\hypertarget{model}{%
\section{Model}\label{model}}

The matters of race and ethnicity are bound to have a very pronounced
impact on the outcome of the 2020 elections due to the cases of racial
injustice and the protests against them which have been held in the
recent past. In such a charged environment, political ideals are
strengthened and their impact on the distribution of votes also
increases. Age and educational levels of voters also have a significant
impact on the outcome of the vote. As we would expect, household income
determines a person's choice in voting since it reflects a person's
satisfaction with the status quo of governance. These factors had
greatly pronounced impacts on the 2016 election and in the feedback
obtained from the survey. As such, we came up with a model which would
predict the results of the presidential election in 2020. It used these
factors as the variables which affect a voter's choice of presidential
candidate as follows(Collier 2000).

\[vote\_trump ~ income + age + edu + ideal + race + \varepsilon\]

\hypertarget{model-specifications}{%
\subsection{Model Specifications}\label{model-specifications}}

After downloading and unzipping the files, the dataset was cleaned to
eliminate such rows as the voters who were not registered since they
would not be allowed to vote. The cleaning process also assigned numeric
values to the educational level, the political ideals, to facilitate the
fitting of the model. The cleaning process was conducted in order to
reduce the dataset to only include the essential information. This would
facilitate the creation of a model from the data since the factors being
used were converted into numbers(Group 2020).\\
The model can be summarised as shown below. The logistic model was
preferred since we aimed at getting a binary output. The output would
simply state whether President Trump would win the election or lose,
presumably to Joe Biden(Larsen et al. 2000).

\begin{Shaded}
\begin{Highlighting}[]
\NormalTok{logmodel <-}\StringTok{ }\KeywordTok{glm}\NormalTok{(vote_trump }\OperatorTok{~}\StringTok{ }\NormalTok{age }\OperatorTok{+}\StringTok{ }\NormalTok{income }\OperatorTok{+}\StringTok{ }\NormalTok{race, }\DataTypeTok{family =}\NormalTok{ binomial, }
             \DataTypeTok{model =} \OtherTok{TRUE}\NormalTok{, }\DataTypeTok{method =} \StringTok{"glm.fit"}\NormalTok{, }\DataTypeTok{data =}\NormalTok{ newsurvey)}

\NormalTok{broom}\OperatorTok{::}\KeywordTok{tidy}\NormalTok{(logmodel)}
\end{Highlighting}
\end{Shaded}

\hypertarget{a-tibble-4-x-5}{%
\section{A tibble: 4 x 5}\label{a-tibble-4-x-5}}

term estimate std.error statistic p.value 1 (Intercept) -2.08 0.136
-15.2 2.16e-52 2 age 0.0103 0.00205 5.04 4.64e- 7 3 income 0.00000479
0.000000800 5.98 2.20e- 9 4 race 1.18 0.0946 12.5 6.23e-36

The model was then fitted to the census data using post stratification
in order to predict the outcome of the elections. The result was the
probability of Donald Trump winning the presidential elections(Pacas and
Sobek, n.d.).

\begin{Shaded}
\begin{Highlighting}[]
\NormalTok{censusdata}\OperatorTok{$}\NormalTok{estimate <-}
\StringTok{  }\NormalTok{logmodel }\OperatorTok
\StringTok{  }\KeywordTok{predict}\NormalTok{(}\DataTypeTok{newdata =}\NormalTok{ censusdata, }\DataTypeTok{type =} \StringTok{"response"}\NormalTok{)}

\NormalTok{prediction <-}\StringTok{ }\KeywordTok{mean}\NormalTok{(censusdata}\OperatorTok{$}\NormalTok{estimate)}
\end{Highlighting}
\end{Shaded}

\hypertarget{results}{%
\section{Results}\label{results}}

The model predicted that the probability of a person voting for
President Trump was: 0.61; regardless of his or her age, income, or
race.

\hypertarget{discussion}{%
\section{Discussion}\label{discussion}}

We conducted model diagnostics with an ANOVA test which gave the
following results:

\begin{Shaded}
\begin{Highlighting}[]
\KeywordTok{anova}\NormalTok{(logmodel, }\DataTypeTok{test=}\StringTok{"Chisq"}\NormalTok{)}
\end{Highlighting}
\end{Shaded}

Analysis of Deviance Table

Model: binomial, link: logit

Response: vote\_trump

Terms added sequentially (first to last)

\begin{verbatim}
   Df Deviance Resid. Df Resid. Dev  Pr(>Chi)    
\end{verbatim}

NULL 4050 5534.0\\
age 1 52.029 4049 5482.0 5.470e-13 \textbf{\emph{ income 1 54.204 4048
5427.8 1.807e-13 }} race 1 177.694 4047 5250.1 \textless{} 2.2e-16 ***
--- Signif. codes: 0 `\emph{\textbf{' 0.001 '}' 0.01 '}' 0.05 `.' 0.1 '
' 1

The analysis of variance showed a large difference between the null
deviance and the residual deviance. The variable of \texttt{race} showed
a significant contribution to the deviance, followed by \texttt{income}
and \texttt{age} respectively. This is because race had the greatest
impact on the probability of any person voting for trump. This means
that race is the most significant factor in swaying people's votes,
followed by the other two factors in the respective order.

\hypertarget{weaknesses}{%
\subsection{Weaknesses}\label{weaknesses}}

It is important to note one weakness of the model as the lack of the
dataset about people's political ideals in the census data. Regardless,
the use of political ideals as perceived by oneself is not an accurate
way to determine the values which a person believes in. This is because
people are susceptible to seeing a false image of themselves.

\hypertarget{next-steps}{%
\subsection{Next Steps}\label{next-steps}}

Besides compensating for the weakness of this model, studies and models
created to determine the potential winner of the United States
Presidential election should accommodate more factors. Although this
model was prudent in its methodology and in the way it backed its
arguments. As such, the models created in future for the same purpose
should be, at least, of the same classification. That is, they should be
logistic regression models.

This code can be found on github through this link:
\url{https://github.com/Juyuhang/Problemset-3}

\newpage

\hypertarget{references}{%
\section*{References}\label{references}}
\addcontentsline{toc}{section}{References}

\hypertarget{refs}{}
\leavevmode\hypertarget{ref-collier2000}{}%
Collier, Paul. 2000. ``Ethnicity, Politics and Economic Performance.''
\emph{Economics \& Politics} 12 (3): 225--45.

\leavevmode\hypertarget{ref-vsg2020}{}%
Group, Voter Study. 2020. ``Nationscape Data Set.''
\url{https://www.voterstudygroup.org/publication/nationscape-data-set}.

\leavevmode\hypertarget{ref-larsen2000}{}%
Larsen, Klaus, Jørgen Holm Petersen, Esben Budtz-Jørgensen, and Lars
Endahl. 2000. ``Interpreting Parameters in the Logistic Regression Model
with Random Effects.'' \emph{Biometrics} 56 (3): 909--14.

\leavevmode\hypertarget{ref-Steven2020}{}%
Pacas, Steven Ruggles; Sarah Flood; Ronald Goeken; Josiah Grover; Erin
Meyer; Jose, and Matthew Sobek. n.d. ``IPUMS Usa: Version 10.0
{[}Dataset{]} Minneapolis, Mn: IPUMS, 2020.''
\url{https://doi.org/https://doi.org/10.18128/D010.V10.0}.

\end{document}
